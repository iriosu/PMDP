\documentclass[11pt, oneside]{article}

% General packages
\usepackage{amsmath, amsfonts, amsthm,amssymb, mathrsfs}
\usepackage{enumitem}
\usepackage[pdftex]{graphicx}
\usepackage[margin=1in]{geometry}
\usepackage{float}
\usepackage{booktabs}
\usepackage{bbm}
\usepackage{caption}
\usepackage{subcaption}
\usepackage{mathtools}
\usepackage{algorithm}
\usepackage{algpseudocode}
\usepackage{setspace}
\usepackage{varwidth}
\usepackage{multirow}
\usepackage{rotating}
\usepackage{array}
\usepackage{tikz}
\usetikzlibrary{arrows}
\usepackage[titletoc]{appendix}
\usepackage{breakcites}
\usepackage{authblk}
\usepackage{mathtools}
\usepackage{dsfont}
\usepackage{makecell}
\usepackage{sansmath}
\usepackage{scalerel}
%\usepackage[round]{natbib}




% Special colors
\definecolor{mygreen}{RGB}{0,128,0}
\renewcommand\Authfont{\fontsize{12}{14.4}\selectfont}
\renewcommand\Affilfont{\fontsize{9}{10.8}\itshape}

% Algorithm configuration
\algnewcommand\algorithmicinput{\textbf{INPUT:}}
\algnewcommand\INPUT{\item[\algorithmicinput]}
\algrenewcommand\Return{\State \algorithmicreturn{} }%

\newcommand{\algorithmicbreak}{\textbf{break}}
\newcommand{\BREAK}{\STATE \algorithmicbreak}

\algtext*{EndWhile}% Remove "end while" text
\algtext*{EndIf}% Remove "end if" text
\algtext*{EndFor}% Remove "end for" text

% Some math stuff
\DeclarePairedDelimiter{\ceil}{\lceil}{\rceil}
\DeclarePairedDelimiter{\floor}{\lfloor}{\rfloor}
\theoremstyle{definition}
%\newtheorem{defn}{Definition}[chapter]
%\newtheorem{exmp}{Example}[chapter]
%\newtheorem{lemma}{Lemma}[chapter]
%\newtheorem{proposition}{Proposition}[chapter]
%\newtheorem{thm}{Theorem}[chapter]

\theoremstyle{plain}
\newtheorem{thm}{Theorem}[section] % reset theorem numbering for each chapter

\theoremstyle{definition}
\newtheorem{defn}[thm]{Definition} % definition numbers are dependent on theorem numbers
\newtheorem{exmp}[thm]{Example} % same for example numbers
\newtheorem{prop}[thm]{Proposition} % definition numbers are dependent on theorem numbers
\newtheorem{coro}[thm]{Corollary} % definition numbers are dependent on theorem numbers
\newtheorem{lema}[thm]{Lemma} % definition numbers are dependent on theorem numbers
\newtheorem{remk}{Remark} % definition numbers are dependent on theorem numbers
\newtheorem{asmp}{Assumption} % definition numbers are dependent on theorem numbers

% Page configuration
\setlength{\parindent}{0.0in}
\setlength{\parskip}{0.08in}

\setlength{\textheight}{8.5in}
\setlength{\textwidth}{5.9in} %.7
\setlength{\topmargin}{-.48in}
\setlength{\oddsidemargin}{.1in} %.29



\newcommand{\R}{\mathbb{R}}
\newcommand{\Z}{\mathbb{Z}}
\newcommand{\N}{\mathbb{N}}
\newcommand{\Q}{\mathbb{Q}}

\newcommand{\M}{\mathcal{M}}
\newcommand{\B}{\mathcal{B}}
\newcommand{\Lagr}{\mathcal{L}}
\newcommand{\I}{\mathcal{I}}
\newcommand{\J}{\mathcal{J}}
\newcommand{\C}{\mathcal{C}}
\newcommand{\D}{\mathcal{D}}


\newcommand{\lotteries}{\mathcal{P}}

\newcommand{\F}{\mathbb{F}}
\newcommand{\Prob}{\mathbb{P}}
\newcommand{\E}{\mathbb{E}}
\newcommand{\V}{\mathbb{V}}


\newcommand{\ev}[1]{\E\lrc{#1}}
\newcommand{\var}[1]{Var\lrp{#1}}


\newcommand{\bX}{\mathbf{X}}
\newcommand{\bx}{\mathbf{x}}
\newcommand{\by}{\mathbf{y}}
\newcommand{\bz}{\mathbf{z}}
\newcommand{\bB}{\mathbf{B}}

\newcommand{\hb}{\hat{\beta}}

\newcommand{\be}{\mathbf{e}}
\newcommand{\bi}{\mathbf{i}}
\newcommand{\bI}{\mathbf{I}}
\newcommand{\bM}{\mathbf{M}}
\newcommand{\bP}{\mathbf{P}}


\newcommand{\cX}{\mathcal{X}}
\newcommand{\cN}{\mathcal{N}}
\newcommand{\cF}{\mathcal{F}}
\newcommand{\cC}{\mathcal{C}}
\newcommand{\cM}{\mathcal{M}}
\newcommand{\cB}{\mathcal{B}}
\newcommand{\cL}{\mathcal{L}}
\newcommand{\cA}{\mathcal{A}}
\newcommand{\cP}{\mathcal{P}}
\newcommand{\cG}{\mathcal{G}}

\newcommand{\tx}{\tilde{x}}

\newcommand{\hp}{\hat{p}}
\newcommand{\hf}{\hat{f}}
\newcommand{\bp}{\bar{p}}


\newcommand{\tr}{\intercal}

\newcommand{\htheta}{\hat{\theta}}
\newcommand{\ind}[1]{\mathds{1}_{\lrl{#1}}}

% delimiters
\newcommand{\lrp}[1]{\left(#1\right)}
\newcommand{\lrc}[1]{\left[#1\right]}
\newcommand{\lrl}[1]{\left\{#1\right\}}
\newcommand{\lra}[1]{\left|#1\right|}
\newcommand{\lrn}[1]{\left\lVert#1\right\rVert}

\newcommand{\prob}[1]{P\lrp{#1}}
\newcommand{\inprob}[1]{P_*\lrp{#1}}
\newcommand{\outprob}[1]{P^*\lrp{#1}}

\def\to{\textbf{to }}

\newcommand{\definition}[1]{\begin{defn}#1 \end{defn}}
\newcommand{\theorem}[1]{\begin{thm}#1 \end{thm}}
\newcommand{\proposition}[1]{\begin{prop}#1 \end{prop}}
\newcommand{\corollary}[1]{\begin{coro}#1 \end{coro}}
\newcommand{\lemma}[1]{\begin{lema}#1 \end{lema}}
\newcommand{\remark}[1]{\begin{remk}#1 \end{remk}}
\newcommand{\assumption}[1]{\begin{asmp}#1 \end{asmp}}
\newcommand{\example}[1]{\begin{exmp}#1 \end{exmp}}

\newcommand{\matrices}[2]{\mathcal{M}_{#1 \times #2}}


\newcommand{\iif}{\Leftrightarrow}
\newcommand{\testhyp}{$H_0: \; \theta \in \Theta_0$ vs. $H_1: \; \theta \in \Theta_0^c$ }
\newcommand{\derpar}[2]{\frac{\partial #1}{\partial #2}}
\newcommand{\parder}[2]{\frac{\partial #1}{\partial #2}}

\newcommand{\limit}[3]{\lim_{#1 \rightarrow #2} #3}
\newcommand{\eqsp}[1]{\begin{equation}\nonumber \begin{split} #1 \end{split}\end{equation}}

\newcommand{\vect}[1]{\lrp{\begin{array}{c} #1 \end{array}}}
\newcommand{\mxtwo}[1]{\lrc{\begin{array}{cc} #1 \end{array}}}
\newcommand{\mxthree}[1]{\lrc{\begin{array}{ccc} #1 \end{array}}}
\newcommand{\piecewise}[1]{\left\{\begin{array}{ll} #1 \end{array}\right.}


\newcommand\norm[1]{\left\lVert#1\right\rVert}
\newcommand{\cupdot}{\mathbin{\mathaccent\cdot\cup}}


\newcommand\ubar[1]{\underline{#1}}

\newcommand{\corr}{\rightrightarrows}


\newcommand{\suminf}[1]{\sum_{#1}^\infty}
\newcommand{\cupinf}[1]{\bigcup_{#1}^\infty}
\newcommand{\capinf}[1]{\bigcap_{#1}^\infty}

\newcommand{\sumn}[2]{\sum_{#1}^{#2}}
\newcommand{\cupn}[2]{\bigcup_{#1}^{#2}}
\newcommand{\capn}[2]{\bigcap_{#1}^{#2}}

\newcommand{\salg}{\sigma-\text{algebra}}
\newcommand{\sfield}{\sigma-\text{field}}

\newcommand{\lsup}[2]{\limsup\limits_{#1\rightarrow #2}}
\newcommand{\linf}[2]{\liminf\limits_{#1\rightarrow #2}}

\usepackage{framed}
\usepackage{makecell}
\usepackage[round]{natbib}   % omit 'round' option if you prefer square brackets

\newcommand{\stcomp}[1]{ {#1}^{\mathsf{c}} }


\bibliographystyle{plainnat}

\title{Documentation Code}
\author{Ignacio Rios \\ Stanford GSB}

\date{\today}

% Document starts here!
\begin{document}
\maketitle
\section{Introduction}
This repository contains files to simulate results for the paper \emph{Procurement
Mechanisms for Differentiated Products}, by Daniela Saban and Gabriel Weintraub.

I implemented it in the programming language Julia, because of its capabilities to
formulate and solve optimization problems. In particular, I use the library JuMP,
which makes the formulation of problems much easier. In addition, to have Julia and
JuMP, the other requirements are installing the solvers Gurobi and Knitro:
\begin{itemize}
    \item Gurobi: general solver for linear and mixed integer problems. I use it to solve the centralized
    problem. It can be easily installed from the Gurobi website, and they have academic
    licenses for free. Once you install it, you have to also install a Julia package to
    use it.
    \item Knitro: general solver for non-linear problems. I use it to solve the decentralized
    problem. This solver can also be installed and there is a trial license for free (which lasts
    around 6 months). An important feature of this solver is the number of starting points
    used during the optimization, since this helps to avoid local optimal solutions.
    So far I implemented it with 500 starting points and it is working well with that;
    however, it takes some time to solve. You may want to decrease that to speed up,
    but you may get worst results.
\end{itemize}

\section{Repository}

The repository Julia contains four main files:
\begin{itemize}
    \item HM.jl: implementation of the Hotelling model.
    \item LDM.jl: implementation of the general affine demand model.
    \item SimulationsHM.jl: implementation to simulate different instances of the Hotelling model.
    \item SimulationsLM.jl: implementation to simulate different instances of the general affine demand model.
    \item utilities.jl: contains a series of methods to compute demands, virtual costs, and check assumptions.
    \item procesingresults.jl: contains a series of methods to process the results of the simulations.
    \item exmaples.jl: tests solutions with close forms to contrast with the results from optimization.
\end{itemize}

In what follows I describe some details that are relevant in each file.

\section{HM.jl}
This file implements the Hotelling demand model. Given \(\Theta = \lrp{\theta_1, \ldots, \theta_m}\) and \(n\) suppliers, the structure of the variables
is
\[
x(\theta) = \lrc{x_1(\theta_1), \ldots, x_n(\theta_1), x_1(\theta_2), \ldots, x_n(\theta_2), \ldots, x_1(\theta_m), \ldots, x_n(\theta_m)}.
\]
The same holds for the other variables, i.e. \(t, p\), and also for the dual variables \(u\) and \(v\).

The formulation of the centralized problem follows directly from the paper, considering as objective function
\[
\max \E_{\theta}\lrc{\sum_{i=1}^n
-\frac{\delta}{2}\lrc{ \lrp{l_i-\sum_{j=1}^{i-1} x_j(\theta)}^2 + \lrp{\sum_{j=1}^{i} x_j(\theta) - l_i}^2  } }
\]
In the case with two suppliers in the extremes of [0,1] this reduces to
\[
\max \E_{\theta}\lrc{
-\frac{\delta}{2}\lrc{ x_1(\theta)^2 + x_2(\theta)^2} - t_1(\theta) - t_2(\theta) }
\]
To implement the decentralized problem, we incorporate the KKT conditions as
additional constraints to the centralized problem. In the two-supplier case these are: for each \(\theta \in \Theta\) and \(i\in \lrl{1,2}\)
\[
\begin{split}
    p_i(\theta) + \delta x_i(\theta) - u_i(\theta) + v(\theta) &= 0 \\
    x_i(\theta)\cdot u_i(\theta) &= 0 \\
    u_i(\theta) & \geq 0 \\
    t_i(\theta) - x_i(\theta)\cdot p_i(\theta) &= 0
\end{split}
\]
Primal feasibility constraints are omitted because they are part of the centralized problem.

This file has the following methods:
\begin{itemize}
    \item GenerateInputs: returns the matrices and vectors to construct the
    constraints and objective of the optimization problem. Some special elements
    of this method are:
    \begin{itemize}
        \item line 12: generates the set of types \(\Theta\), i.e. all possible
        combinations of types of suppliers.
        \item line 27-34: represents computation of \(f_{-i}(\theta_{-i})\).
    \end{itemize}
    The outputs of this method are:
    \begin{itemize}
        \item nsupp: number of suppliers
        \item ntypes: number of types for each supplier;
        \item nvars: length of each vector of variables, i.e. \(n\times m\)
        \item sts: number of scenarios, i.e. \(\lra{\Theta}\)
        \item bGx: matrix multiplying \(x\) in the inequality IC and IR constraints.
        This matrix is organized as follows:
        \begin{itemize}
            \item The first nsupp \(\times\) ntypes rows are IR constraints, ordered
            lexicographically by \((i, \theta_i)\) in increasing order, where \(i\) is the supplier
            and \(\theta_i\) is his type. For example, in a 2 supplier-2 types
            example, the first row corresponds to supplier 1 with low type, then
            supplier 1 with high type, then supplier 2 with low type, and
            finally supplier 2 with high type.
            \item The last nsupp \(\times\) ntypes \(\times\) (ntypes-1) rows correspond
            to IC constraints, ordered lexicographically by \((i, \theta_i, \theta_i')\) in
            increasing order.
        \end{itemize}
        \item bGt: matrix multiplying \(t\) in the inequality IC and IR constraints
        \item bh: right-hand side of IR-IC constraints (equal to a vector of zeros)
        \item bA: matrix multiplying \(x\) in the feasibility constraints
        \item bb: right-hand side of feasibility constraints (equal to a vector of ones)
        \item wqt: vector with the probabilities of each scenario in \(\Theta\) to compute the expected value for objective function
        \item f: joint distribution of types
        \item Theta: equivalent to \(\Theta\), i.e. the set of all types.
    \end{itemize}
    \item GenerateExPostInputs: receives as input the list of types, the set of type combinations, the number of suppliers and the number of variables. Generates as output the matrices and vectors required
    to write ex-post IR constraints.
    \item CheckFeasibility: receives the inputs of the problem and solutions
    \(x_0, t_0\) and checks whether this solution is feasible for the decentralized problem. To accomplish this, I fix \(x=x_0, t=t_0\) and solve the decentralized problem with objective value equal to 1.
    \item SolveOptimization: formulates and solves the optimization problem using as inputs the parameters of the problem and version, which is equal to centralized or decentralized.
\end{itemize}


\section{LDM.jl}
This file implements the general affine linear demand model. The structure of the variables is equivalent to the previous case. The centralized problem
follows directly from the paper, while the decentralized problem is implemented
by adding the following KKT conditions as constraints (in matrix form):
\[
\begin{split}
    p(\theta) - c + Dx(\theta) - u(\theta) + v(\theta) &= 0 \\
    x(\theta)\cdot u(\theta) &= 0 \\
    u(\theta) & \geq 0 \\
    t(\theta) - x(\theta)\cdot p(\theta) &= 0
\end{split}
\]
The methods in this file are the same as in the previous case adapted to the general affine demand model.
In particular, these methods are:
\begin{itemize}
  \item GenerateInputs: same as before. Actually, these methods are exactly equivalent to that in the Hotelling case.
  \item GenerateExPostInputs: same as before. Actually, these methods are exactly equivalent to that in the Hotelling case.
  \item InputsObjectiveLM: these method generates special inputs for the general affine demand model. In particular, it computes
  \[
  D = \Gamma^{-1}, \quad c = D \alpha
  \]
  and also generates the inputs for the objective function of the affine model.
  \item CheckFeasibility: given a solution to the affine model, it checks whether the solution is feasible.
  \item SolveOptimization: formulates and solves the optimization problem using as inputs the parameters of the problem and version, which is equal to centralized or decentralized.
  In addition, it receives the following additional inputs:
  \begin{itemize}
    \item elastic: boolean variable that is true if we want to solve elastic model, and false if we want to solve inelastic model.
    \item expostir: boolean variable that is true if we want to incorporate ex-post IR constraints, and false if we want to consider only interim constraints.
    \item x0: vector with initial solution - allocation
    \item t0: vector with initial solution - transfers
  \end{itemize}
  Note: the solvers considered are two:
  \begin{itemize}
    \item Gurobi: used to solve the centralized problem.
    \item Knitro: used to solve the decentralized problem. This solver is needed because the decentralized problem
    is not linear, so Gurobi cannot handle it. When setting the solver, one parameter that is relevant is
    \textbf{msmaxsolves}, which is the number of multi-start points considered.
    Considering several starting points is important to avoid local optima solutions
    that are not globally optimal.
  \end{itemize}
  \item FormulateAndSolve: considers the inputs of the problem to solve both the centralized and
  the decentralized models, and returns:
  \begin{itemize}
    \item objective function of the centralized model
    \item expected transfer of the centralized model
    \item vector of allocation of the centralized model, i.e. \(x(\theta), \; \theta \in \Theta\)
    \item vector of transfers of the centralized model, i.e. \(t(\theta), \; \theta \in \Theta\)
    \item objective function of the decentralized model
    \item expected transfer of the decentralized model
    \item vector of allocation of the decentralized model
    \item vector of transfers of the decentralized model
  \end{itemize}
\end{itemize}

\section{SimulationsHM.jl}
The implementation of the Hotelling model considers only two suppliers and any potential
number of types. Implementing more suppliers is tricky because the the KKT
conditions are messy.

The methods considered in this file are:
\begin{itemize}
  \item SimulateTwoSuppliers: this method solves the Hotelling model for a list of different \(\delta\).
  The inputs are:
  \begin{itemize}
    \item Types: list of types \(\Theta_i\)
    \item Dictionary of marginal distributions
    \item Vector of locations for Hotelling model
    \item Vector of \(\delta\)s to perform simulation.
    \item elastic: boolean that is true for elastic demand, false for inelastic.
    \item expostir: boolean that is true for expost IR constraints, false for including only interim IR constraints.
  \end{itemize}

  The output generated by this method is a file that contains, for each \(\delta\),
  \begin{itemize}
    \item value of \(\delta\) considered
    \item objective function of the centralized model
    \item expected transfer of the centralized model
    \item vector of allocation of the centralized model, i.e. \(x(\theta), \; \theta \in \Theta\)
    \item vector of transfers of the centralized model, i.e. \(t(\theta), \; \theta \in \Theta\)
    \item objective function of the decentralized model
    \item expected transfer of the decentralized model
    \item vector of allocation of the decentralized model
    \item vector of transfers of the decentralized model
  \end{itemize}
\end{itemize}

The execution starts in line 44. There you can write the parameters related to the
types of suppliers, the distribution of types, the parameters for transportation
cost and location of suppliers, as well as what we want to vary during simulation.

\section{SimulationsLM.jl}
For the general affine demand case, we can solve the problem for different number of
supplier and for several number of types. The methods included in this file are:
\begin{itemize}
    \item SimulateTwoSuppliersAsymmetric: this method solves the problem for both
    the centralized and decentralized cases considering two suppliers, for different parameters of quality and
    demand functions. This method considers all possible combinations of parameters provided as inputs. The inputs
    are:
    \begin{itemize}
        \item types: vector of types for each supplier \(\Theta_i\)
        \item fm: marginal distributions of types
        \item aj: list of potential values for the quality of suppliers
        \item gammaii: list of potential values for the elements on the diagonal of the matrix \(\Gamma\)
        \item gammaij: list of potential values for the elements off the diagonal of the matrix \(\Gamma\)
        \item elastic: boolean that is true for elastic demand, false for inelastic
        \item expostir: boolean that is true for expost IR constraints, false for including only interim IR constraints
    \end{itemize}

    \item SimulateNSupplierSymmetric: this method solves the symmetric problem for any number of
    suppliers and types, i.e. it assumes that all suppliers provide the same quality and that the
    demand matrix is symmetric and all elements in its diagonal are the same. The inputs for this
    function are the same as before, with an additional one, \(N\), which is the desired number of
    suppliers.
\end{itemize}

\section{utilities.jl}
This file contains a series of methods that are used in the previous files.
These methods are:
\begin{itemize}
    \item combine and combwithrep: used to generate all potential profiles in \(\Theta\).
    \item ComputeCumulativeDistribution: for a given dictionary of marginal
    distributions, returns a disctionary of marginal cumulative distributions.
    \item virtualcost: computes, for a given supplier \(i\) and type \(\theta\), the virtual cost.
    \item demandLM: computes the demand faced by supplier \(i\) in the general
    affine model with two suppliers (obtained from close form solution).
    \item assortmentHM: returns the set of suppliers in the assortment of the
    Hotelling model for a given price and \(\delta\).
    \item demandHM: computes the demand faced by supplier \(i\) in the
    Hoteeling model with two suppliers.
    \item checkvcincreasing: checks whether the virtual costs are increasing
    for each supplier.
    \item checkconditionsHM: check whether the conditions in Theorem 4.1 are
    satisfied; returns a list with two booleans, one for each condition.
    \item checkconditionsLM: check whether the conditions in Theorem 5.2 are
    satisfied; returns a list with two booleans, one for each condition. To
    find \(d^*\), we use the following heuristic:
    \begin{enumerate}
        \item Divide \(\Theta\) in two subsets \(\Theta^S\) and \(\Theta^N\),
        such that
        \[
        \begin{split}
        \Theta^S &= \lrl{\theta \in \Theta: \; x_i(\theta) > 0, \; \forall i} \\
        \Theta^N &= \lrl{\theta \in \Theta: \; \exists i \; st.\;  x_i(\theta) = 0}
        \end{split}
        \]
        \item For each \(\theta \in \Theta^S\), compute
        \[
        d(\theta) = \max\lrl{\lra{v_i(\theta)-v_i(\theta')}: \theta' \in \Theta^N, \; i=1, \ldots, n}
        \]
        \item Define
        \[
        d^* = \min\lrl{d(\theta):\; \theta\in \Theta^S}
        \]
    \end{enumerate}

\end{itemize}

\section{processingresults.jl}
This file contains function that allow to read the outputs generated during simulation,
and analyze different statistics or results.

\section{examples.jl}
This file tests the methods described previously.



\end{document}
